% !TEX root = ../Transactions_report.tex
\section{Выводы}

В ходе выполнения работы было изучено создание и управление транзакциями, а также на примерах рассмотрены уровни изоляции. 

Транзакции позволяют выполнять контроль целостности данных при выполнении различных запросов, но если требовать последовательной обработки транзакций, то возможно уменьшение производительности, т.к. при параллельном выполнении транзакций возможны следующие проблемы:

\begin{itemize}
	\item Грязное чтение (Dirty Read) --- чтение данных, добавленных или измененных транзакцией, которая впоследствии не подтвердится (откатится);
	\item Размытое чтение (Fuzzy Read) --- при повторном чтении в рамках одной транзакции, ранее прочитанные данные оказываются измененными;
	\item Фантомное чтение (Phantom Read) --- одна транзакция в ходе своего выполнения несколько раз выбирает множество строк по одним и тем же критериям, другая транзакция в интервалах между этими выборками добавляет или удаляет строки или изменяет столбцы некоторых строк, используемых в критериях выборки первой транзакции, и успешно заканчивается; в результате получится, что одни и те же выборки в первой транзакции дают разные множества строк.
\end{itemize}

Для достижения компромисса между быстродействием существуют несколько уровней изоляции транзакций. При этом можно выделить соответствие между уровнями изоляции и возможными конфликтами:

\begin{table}[H]
	\centering
	\begin{tabular}{|c|c|c|c|}
		\hline \textbf{Уровень изоляции} & \textbf{Dirty Read} & \textbf{Fuzzy Read} & \textbf{Phantom Read}      	\\
		\hline READ UNCOMMITTED          & Возможно            & Возможно            & Возможно         \\
		\hline READ COMMITED             & Невозможно          & Возможно            & Возможно         \\
		\hline REPEATABLE READ           & Невозможно          & Невозможно          & Возможно         \\
		\hline SERIALIZABLE              & Невозможно          & Невозможно          & Невозможно       \\
		\hline		
	\end{tabular}
\end{table}

Таким образом, сериализуемость --- это способность к  упорядочению параллельной обработки транзакций. Также можно установить соответствие между уровнями изоляции по SQL-стандарту и уровнями изоляции транзакций в Firebird SQL: 

\begin{table}[H]
	\centering
	\begin{tabular}{|c|c|c|}
		\hline \textbf{Уровни изоляции по SQL} & \textbf{Уровни изоляции транзакций в Firebird}    \\
		\hline READ COMMITED               & READ COMMITED                          \\
		\hline REPEATABLE READ             & SNAPSHOT                               \\
		\hline SERIALIZABLE                & SNAPSHOT TABLE STABILITY               \\
		\hline		
	\end{tabular}
\end{table}