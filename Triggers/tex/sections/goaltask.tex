% !TEX root = ../Triggers_report.tex
\section{Цель работы}

Познакомиться с возможностями реализации более сложной обработки данных на стороне сервера с помощью хранимых процедур и триггеров.

\section{Программа работы}

\begin{enumerate}
	\item Создать два триггера: один триггер для автоматического заполнения ключевого поля, второй триггер для контроля целостности данных в подчиненной таблице при удалении/изменении записей в главной таблице;
	\item Создать триггер в соответствии с индивидуальным заданием, полученным у преподавателя;
	\item Создать триггер в соответствии с индивидуальным заданием, вызывающий хранимую процедуру;
	\item Выложить скрипт с созданными сущностями в системе контроля версий;
	\item Продемонстрировать результаты преподавателю.
\end{enumerate}

\section{Индивидуальные задания}

\begin{enumerate}
	\item Реализовать триггеры, не позволяющие несколько раз добавить трек в плейлист и альбом.
	\item При добавлении новых данных в таблицу продаж альбомов запускать пересчет агрегатов по окончившемуся периоду: если последняя запись в таблицу продаж была добавлена до окончания очередного агрегационного этапа, а добавляемая --- после, то запускать процедуру пересчета для окончившегося периода.
\end{enumerate}