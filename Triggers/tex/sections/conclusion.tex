% !TEX root = ../Triggers_report.tex
\section{Выводы}

В ходе выполнения данной лабораторной работы было проведено ознакомление с возможностями реализации более сложной обработки данных на стороне сервера с помощью триггеров, разработаны триггеры автоматического заполнения первичного ключа, контроля целостности БД, остановки добавления записи в таблицу по исключению и и триггер, вызывающий процедуру.

Основной областью применения триггеров является контроль целостности базы данных любой сложности. С помощью триггеров существенно упрощаются приложения, т.к. часть логики приложения автоматически активизируется при обновлении БД и выполняется на стороне сервера.

К недостаткам триггеров можно отнести сложность разработки и поддержки, а также скрытая от пользователя функциональность, что может привести к незапланированным последствиям, причины которых потенциально непросто обнаружить. Триггеры влияют на производительность системы, т.к. перед выполнением каждой команды по изменению состояния БД необходимо проверять необходимость запуска триггера.

Таким образом, триггеры --- очень полезный инструмент разработки БД, требующий аккуратного к нему подхода.