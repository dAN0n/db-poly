% !TEX root = ../DML_report.tex
\section{Цель работы}

Познакомиться с языком создания запросов управления данными SQL-DML.

\section{Программа работы}

\begin{enumerate}
	\item Изучить SQL-DML;
	\item Выполнить все запросы из списка стандартных запросов. Продемонстрировать результаты преподавателю;
	\item Получить у преподавателя и реализовать SQL-запросы в соответствии с индивидуальным заданием. Продемонстрировать результаты преподавателю;
	\item Выполненные запросы SELECT сохранить в БД в виде представлений, запросы INSERT, UPDATE или DELETE --- в виде ХП.
\end{enumerate}

\section{Список стандартных запросов}

\begin{enumerate}
	\item Сделать выборку всех данных из каждой таблицы;
	\item Сделать выборку данных из одной таблицы при нескольких условиях, с использованием логических операций LIKE, BETWEEN, IN (не менее 3-х разных примеров);
	\item Создать в запросе вычисляемое поле;
	\item Сделать выборку всех данных с сортировкой по нескольким полям;
	\item Создать запрос, вычисляющий несколько совокупных характеристик таблиц;
	\item Сделать выборку данных из связанных таблиц (не менее двух примеров);
	\item Создать запрос, рассчитывающий совокупную характеристику с использованием группировки, наложите ограничение на результат группировки;
	\item Придумать и реализовать пример использования вложенного запроса;
	\item С помощью оператора INSERT добавить в каждую таблицу по одной записи;
	\item С помощью оператора UPDATE изменить значения нескольких полей у всех записей, отвечающих заданному условию;
	\item С помощью оператора DELETE удалить запись, имеющую максимальное (минимальное) значение некоторой совокупной характеристики;
	\item С помощью оператора DELETE удалить записи в главной таблице, на которые не ссылается подчиненная таблица (используя вложенный запрос).
\end{enumerate}