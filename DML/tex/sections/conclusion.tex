% !TEX root = ../DML_report.tex
\section{Выводы}

В ходе выполнения данной работы было проведено ознакомление с языком SQL-DML, позволяющий производить операции над данными в БД. К основным операциям SQL-DML можно отнести:

\begin{itemize}
 	\item INSERT --- добавление записи;
 	\item SELECT --- выборка данных;
 	\item DELETE --- удаление записей;
 	\item UPDATE --- обновление значений записей;
 	\item JOIN --- объединение таблиц для выборки;
 	\item WHERE --- условия отбора в выборку;
 	\item ORDER BY --- группировка выбоки.
 \end{itemize} 

Использование данных операторов позволяет производить широкий круг действий над БД, таких как выборка данных, объединение таблиц, вложенные запросы, наложение условий для выборки и т.д. 

Использование представлений позволяет сохранить частые запросы в БД как виртуальные таблицы для более простого их вызова. Хранимые процедуры являются аналогом функций в языках программирования и позволяют сохранить часто используемые однотипные операции сложной выборки с использованием переменных в БД. Данные инструменты упрощают выполнение рутинных действий с базой данных.