% !TEX root = ../SP_report.tex
\section{Ход работы}

\subsection{Реализовать процедуру выдачи награды за самый популярный по продажам альбом: за заданный период вычислять самый продаваемый альбом и выдавать ему награду заданного типа}

\lstinputlisting[language=SQL, morekeywords={REFERENCES, to, PROCEDURE, DECLARE, VARIABLE, BEGIN}]{code/ind1.sql}

\subsection{Реализовать процедуру вычисления агрегатов --- сумму продаж по дням, неделям, месяцам, годам, городам, странам. Для хранения вычисленных значений необходимо создать отдельную таблицу.}

Для хранения вычисленных значений создадим по таблице для каждого агрегата:

\lstinputlisting[language=SQL, morekeywords={REFERENCES, to, PROCEDURE, BEGIN, DECLARE, VARIABLE, IF, FOR, DO, SUSPEND}]{code/ind2_t.sql}

Реализованная процедура:

\lstinputlisting[language=SQL, morekeywords={REFERENCES, to, PROCEDURE, BEGIN, DECLARE, VARIABLE, IF, FOR, DO, SUSPEND, DATEADD, MATCHING}]{code/ind2.sql}