% !TEX root = ../SP_report.tex
\section{Выводы}

В ходе выполнения данной работы было проведено ознакомление с хранимыми процедурами, хранящимися в базе данных и напоминающие функции из языков программирования. Хранимые процедуры позволяют упростить выполнение однотипных операций сложной выборки данных из базы, например, в случае отличия только в константах.

Использование хранимых процедур повышает безопасность, позволяя давать пользователю доступ только к ним и изолировать от самой базы данных, уменьшает количество запросов к серверу, вследствие чего повышается скорость работы. 

К недостаткам хранимых процедур можно отнести отсутствие контроля целостности кода при изменении структуры БД, вследствие чего поддержка такой БД более затратна, "`размазывание"' логики между клиентским приложением и БД, непереносимость между различными диалектами СУБД.