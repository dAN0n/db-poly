% !TEX root = ../ER_report.tex
\section{Выводы}

В ходе выполнения данной работы было проведено ознакомление с основами проектирования баз данных и разработана SQL-схема, приведенная к 3НФ, в которой наглядно показана структура БД и взаимосвязность таблиц друг с другом посредством первичных и вторичных ключей. Приведение к третьей нормальной форме имеет следующие преимущества:

\begin{itemize}
	\item Уменьшение пространства, занимаемого базой данных;
	\item Поддержка целостности данных при модификации.
\end{itemize}

Однако нормализация имеет и недостаток: для выборки требуемых данных необходимо выполнять более сложные запросы, включающие присоединения нескольких таблиц, что приводит к уменьшению производительности выполняемых запросов. Для решения данной проблемы можно выполнить намеренную денормализацию БД, но потребуется более тщательно контролировать целостность данных.