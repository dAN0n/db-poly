% !TEX root = ../DDL_report.tex
\section{Выводы}

В ходе выполнения данной работы было проведено ознакомление с языком определения данных SQL-DDL, создана и модифицирована структура базы данных с помощью SQL-скриптов, а также выполнено ознакомление с приложением IBExpert. 

SQL-DDL прост и понятен, поэтому освоить его не составляет особого труда. Стоит отметить, что следует быть крайне осторожным с исполнением SQL-скриптов, так как последствия работы ошибочного скрипта может быть непросто исправить.

Приложение IBExpert позволяет работать с несколькими базами данных, используя графический интерфейс, а также предлагает несколько полезных утилит, таких как Database Designer для создания ER-диаграм (к сожалению, без автоматического распределения всех таблиц в удобном для пользователя виде, что может стать проблемой при большом количестве таблиц) и Test Data Generator для автоматической генерации записей для таблиц. Данное приложение заметно упрощает работу с БД.